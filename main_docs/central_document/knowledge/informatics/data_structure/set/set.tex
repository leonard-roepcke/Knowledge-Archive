\documentclass[12pt,a4paper]{article}

% --- Standard-Imports, die fast immer verfügbar sind ---
\usepackage[utf8]{inputenc}     % UTF-8 Unterstützung
\usepackage[T1]{fontenc}        % europäische Zeichen
\usepackage[ngerman]{babel}     % deutsche Silbentrennung
\usepackage{geometry}           % Seitenränder einstellen
\usepackage{graphicx}           % Bilder einbinden
\usepackage{hyperref}           % klickbare Links
\usepackage{xcolor}             % Farben (wird von hyperref genutzt)

% --- Seitenlayout ---
\geometry{margin=2.5cm}

% --- Hyperlink-Stil (wissenschaftlich neutral) ---
\hypersetup{
    colorlinks=true,
    linkcolor=blue,
    urlcolor=blue,
    citecolor=black,
    pdfauthor={Leonard Röpcke},
    pdftitle={pdftitle}
}

\title{Set in bezug auf Python}
\author{Leonard Röpcke}
\date{\today}

\begin{document}
\maketitle
%\tableofcontents
%\newpage

\section{Erklärung}
Ein Set ist eine spezielle Art von HashMap. Es ist eine unsortierte Sammlung, durch die man nicht iterieren kann.
Man kann es sich so vorstellen: Der Key wird gehasht, und der Hashwert dient als Index.
Wenn dieser Index bereits belegt ist, liefert das Set True zurück, andernfalls False.


\end{document}
