\documentclass[12pt,a4paper]{article}

% --- Standard-Imports, die fast immer verfügbar sind ---
\usepackage[utf8]{inputenc}     % UTF-8 Unterstützung
\usepackage[T1]{fontenc}        % europäische Zeichen
\usepackage[ngerman]{babel}     % deutsche Silbentrennung
\usepackage{geometry}           % Seitenränder einstellen
\usepackage{graphicx}           % Bilder einbinden
\usepackage{hyperref}           % klickbare Links
\usepackage{xcolor}             % Farben (wird von hyperref genutzt)
\usepackage{amsmath}

% --- Seitenlayout ---
\geometry{margin=2.5cm}

% --- Hyperlink-Stil (wissenschaftlich neutral) ---
\hypersetup{
    colorlinks=true,
    linkcolor=blue,
    urlcolor=blue,
    citecolor=black,
    pdfauthor={Leonard Röpcke},
    pdftitle={hash_maps}
}

\title{Hash Maps}
\author{Leonard Röpcke}
\date{\today}

\begin{document}
\maketitle
%\tableofcontents
%\newpage

\section*{Erklärung}
Eine Hash Map ist eine erweiterte Form eines 
\href{https://leonard-roepcke.github.io/Knowledge-Archive/knowledge/informatics/data_structure/set/set.pdf}{Sets}
. Während ein 
\href{https://leonard-roepcke.github.io/Knowledge-Archive/knowledge/informatics/data_structure/set/set.pdf}{Set}
 nur einzelne Schlüssel speichert, ordnet eine Hash Map jedem Schlüssel (Key) einen zugehörigen 
 Wert zu. Dazu wird der Key mithilfe einer Hash-Funktion in einen Hash-Wert umgewandelt. 
 Dieser Hash-Wert bestimmt die Position im Speicher, an der der zugehörige Wert abgelegt ist.

\end{document}
