\documentclass[12pt,a4paper]{article}

% --- Standard-Imports, die fast immer verfügbar sind ---
\usepackage[utf8]{inputenc}     % UTF-8 Unterstützung
\usepackage[T1]{fontenc}        % europäische Zeichen
\usepackage[ngerman]{babel}     % deutsche Silbentrennung
\usepackage{geometry}           % Seitenränder einstellen
\usepackage{graphicx}           % Bilder einbinden
\usepackage{hyperref}           % klickbare Links
\usepackage{xcolor}             % Farben (wird von hyperref genutzt)

% --- Seitenlayout ---
\geometry{margin=2.5cm}

% --- Hyperlink-Stil (wissenschaftlich neutral) ---
\hypersetup{
    colorlinks=true,
    linkcolor=blue,
    urlcolor=blue,
    citecolor=black,
    pdfauthor={Leonard Röpcke},
    pdftitle={vsc_shortcuts}
}

\title{VSC Shortcuts}
\author{Leonard Röpcke}
\date{\today}

\begin{document}
\maketitle
%\tableofcontents
%\newpage

\begin{center}
    
    \section*{General Shortcuts}
    \begin{tabular}{|p{6cm}||p{6cm}|}
        \hline
        Befehle & Beschreibung \\
        \hline
        strg + N & Neues Dokument erstellen \\
        \hline
        strg + K strg + o & Ordner in VSC öffnen \\
        \hline
    \end{tabular}
    
    
    \section*{Personal Shortcuts}
    \begin{tabular}{|p{6cm}||p{6cm}|}
        \hline
        Befehle & Beschreibung \\
        \hline
        strg + m, strg + e & Alle Ordner minimieren \\
        \hline
    \end{tabular}
    
\end{center}
    
\end{document}
