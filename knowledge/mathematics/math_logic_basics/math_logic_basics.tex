\documentclass[12pt,a4paper]{article}

% --- Standard-Imports, die fast immer verfügbar sind ---
\usepackage[utf8]{inputenc}     % UTF-8 Unterstützung
\usepackage[T1]{fontenc}        % europäische Zeichen
\usepackage[ngerman]{babel}     % deutsche Silbentrennung
\usepackage{geometry}           % Seitenränder einstellen
\usepackage{graphicx}           % Bilder einbinden
\usepackage{hyperref}           % klickbare Links
\usepackage{xcolor}             % Farben (wird von hyperref genutzt)
\usepackage{amsmath}

% --- Seitenlayout ---
\geometry{margin=2.5cm}

% --- Hyperlink-Stil (wissenschaftlich neutral) ---
\hypersetup{
    colorlinks=true,
    linkcolor=blue,
    urlcolor=blue,
    citecolor=black,
    pdfauthor={Leonard Röpcke},
    pdftitle={math_logic_basics}
}

\title{Mathe Basics}
\author{Leonard Röpcke}
\date{\today}

\begin{document}
\maketitle
\tableofcontents
\newpage

\section{Eine Aussage}

In der Mathematik ist eine \textbf{Aussage} ein Satz, der \textbf{entweder wahr oder falsch} sein kann. 
Zum Beispiel ist der Satz 
\textit{„Es regnet gerade.“} 
eine Aussage, da er eindeutig entweder wahr oder falsch sein kann.

Hingegen ist der Satz 
\textit{„Es ist warm.“} 
\textbf{keine} Aussage, da er nicht eindeutig mit \textit{wahr} oder \textit{falsch} bewertet werden kann 
der Begriff \textit{„warm“} ist subjektiv und nicht klar definiert.

Um Aussagen zu vereinfachen und sie nicht jedes Mal ausschreiben zu müssen, 
definiert man Abkürzungen. Zum Beispiel:
\[
A := \text{Es regnet.}
\]

Damit kann man schreiben:
\[
\text{Wenn } A \text{, dann ist der Boden nass.}
\]


\end{document}
