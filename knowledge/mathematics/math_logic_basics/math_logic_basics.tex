\documentclass[12pt,a4paper]{article}

% --- Standard-Imports, die fast immer verfügbar sind ---
\usepackage[utf8]{inputenc}     % UTF-8 Unterstützung
\usepackage[T1]{fontenc}        % europäische Zeichen
\usepackage[ngerman]{babel}     % deutsche Silbentrennung
\usepackage{geometry}           % Seitenränder einstellen
\usepackage{graphicx}           % Bilder einbinden
\usepackage{hyperref}           % klickbare Links
\usepackage{xcolor}             % Farben (wird von hyperref genutzt)

% --- Seitenlayout ---
\geometry{margin=2.5cm}

% --- Hyperlink-Stil (wissenschaftlich neutral) ---
\hypersetup{
    colorlinks=true,
    linkcolor=blue,
    urlcolor=blue,
    citecolor=black,
    pdfauthor={Leonard Röpcke},
    pdftitle={math_logic_basics}
}

\title{Mathe Basics}
\author{Leonard Röpcke}
\date{\today}

\begin{document}
\maketitle
\tableofcontents
\newpage

\section{Ein Aussage}
In der Mathematik ist eine Aussage ein Satzt der Entweder Richtig oder Falsch sein kann. Zum Beispiel ist der Satz " Es Regnet gerade" eine Aussage.
Hingegen "Es ist Warm" keine da dieser Satz nicht eindeutig mit Ja oder Nein Beantwortet werden kann.
Um Aussagen zu vereinfachen und sie nicht jedes mal neu schreiben zu müssen definiert man folgender massen eine Abkürtzung. A:=Es Regnet.
Das heißt ich kann sagen "Wenn A dann ist der Boden Nass".

\end{document}
